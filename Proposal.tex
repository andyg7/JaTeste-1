\documentclass{article} 
\usepackage{url, graphicx}
\usepackage[margin=1in]{geometry}
\usepackage{textcomp}
\usepackage{algpseudocode}
\usepackage{algorithm}
\usepackage{titling}
\usepackage{amsmath}
\usepackage{amssymb}
\usepackage{amsthm}
\usepackage{verbatim}
\usepackage{listings} % for code highlighting/formatting

\usepackage{color} %defining colors for syntax highlighting
\definecolor{syntaxBlue}{RGB}{42,0.0,255}
\definecolor{syntaxGreen}{RGB}{63,127,95}
\definecolor{syntaxPurple}{RGB}{127,0,85}
\definecolor{syntaxCyan}{RGB}{0,155,155}
\definecolor{syntaxGreyBg}{RGB}{220,220,220}

\lstdefinelanguage{JaTeste} %define the code highlighting/formatting
{
	% list of keywords
	morekeywords={
		func,
		with,
		test,
		if,
		else,
		while,
		for,
		return,
		using
	},
	sensitive=true, % keywords ARE case-sensitive
	morecomment=[s]{/*}{*/}, % /* and */ delimit comments
	morestring=[b]" % string's MUST be in double quotes
}
\lstset{
	language={JaTeste}, % tell listings package to use the JaTeste language spec
	basicstyle=\small\ttfamily, % Global Code Style
	tabsize=2, % number of spaces indented when discovering a tab 
	columns=fixed, % make all characters equal width
	keepspaces=true, % does not ignore spaces to fit width, convert tabs to spaces
	showstringspaces=false, % lets spaces in strings appear as real spaces
	breaklines=true, % wrap lines if they don't fit
	frame=trbl, % draw a frame at the top, right, left and bottom of the listing
	frameround=tttt, % make the frame round at all four corners
	framesep=4pt, % quarter circle size of the round corners
	numbers=left, % show line numbers at the left
	numberstyle=\tiny\ttfamily, % style of the line numbers
	commentstyle=\color{syntaxGreen},
	keywordstyle=\color{syntaxPurple},
	stringstyle=\color{syntaxBlue},
	emph={int,char,float,struct,string},
	emphstyle=\color{syntaxCyan},
	backgroundcolor=\color{syntaxGreyBg},
}

\title{PLT 4115 Proposal}
\author{Andrew Grant, Jemma Losh, Jared Weiss, Jake Weismann  \\amg2215@columbia.edu, jal2285@columbia, jbw2140@columbia.edu, jdw2159@columbia.edu}
\date{2/8/16}

\begin{document}

\maketitle

\section{Description}
As necessitated by a language that assures a programmer of the effectivity of his or her code, our language is designed to be fairly rigid and explicit in structure, without much behind-the-scenes magic that can make some existing programming languages difficult to read and understand. While at first this may seem like a limitation to the programmer who is well versed in non-statically typed languages (i.e Python) or object-oriented languages (i.e. C++ or Java), our language is just as capable, while also ensuring a robust compiled program that should always work as expected. //
The syntax for our language will explicitly define the domain and range (including errors that may be raised) of functions for integration tests, as well as the expected input and output for various cases for unit tests. At compilation time, all of these tests will be run to make sure that the produced program works the way it?s designers intended. In addition to user defined unit tests, we will analyze the input code and generate some of our own tests. These tests will be generated in order to cover important test cases that the user might have missed and cases that ensure full code coverage (i.e. all lines of the input code has been tested).

\subsection{Compiler Proposals}
We are considering two alternatives for the workflow of our compiler. The first option is an interpreter-compiler hybrid that runs unit tests and coverage tests at compile time. This method ensures that a program that does not successfully pass all tests will not have the ability to be compiled, and therefore will not have the chance for execution by the programmer. This would ensure that errand code does not make it through the compiler and into production. On failure, the compiler will inform the user as to what tests failed and what can be done to improve the quality and testability of the program. \\
The second consideration is to generate a test file based on the input program. At compile time, two files will be generated: (1) an executable file, and (2) an executable test file with all the relevant test cases. This method would allow the user to continue with his or her normal work flow and minimize interference from the compiler, while at the same time providing a robust test file to fully test one's program. As compared to the first option, this allows for quicker up times for programmers but would lower the level of rigidity the first option provides.

\subsection{Types}
\begin{itemize}
	\item int
	\item char
	\item float
	\item struct
	\item arrays[] \textit{(of any type)}
\end{itemize}

In our language, typing will be strongly enforced throughout every possible step of compilation. All variables in our language will need to be typed before they can be referenced (and naturally before assignment). This is similar to the strong typing of C, albeit without the complexities that are associated with passing pointers.
\section{Syntax}
While our syntax is inspired by C, it is very much functional. Each function can only rely on variables passed to it explicitly; functions cannot reference global variables or variables outside the functions scope. This ensures functions always return the same value when passed 

\subsection{Comments}

/* No single line comments, only nested comments */

\subsection{Key Words}
\begin{itemize}
	\item
	with test: 
	\item
	using:
	\item 
	func
\end{itemize}

	
\section{Example code}
\begin{lstlisting}
/*
	We are using multi-line comments.
	There are no single line comments.
*/

/* the func keyword is used for function declaration */
func int absoluteValue(int a)
{
	if (a < 0) {
		a = -a;
	} else {
		a = a;
	}
	return a;
	 /* This is how you use the keyword "with test: " . Tests are attached to the end of function definitions with some sort of boolean condition. In the test below, the programmer  makes sure the function absoluteValue returns a positive number when  passed the value -2
	*/

}
with test: absoluteValue(-2) > 0;

func int sum(int[] arr, int len)
{
	int sum;
	int i;
	
	i = 0;
	while ( i < len; i++) {
		sum = sum + arr[i];
	}
	return sum;
}
with test sum(arr, 5) == 15 using int[] arr = {1, 2, 3, 4, 5};
/*
	The "using" keyword above is used to create variables that are used in the test condition.  Here, the programmar created an array using the keyword and makes sure the sum function computes the right value
*/


func int add(int a, int b)
{
	return x + y;
}
with test: add(1,2) == 3
with test: 0 < add(1,2) < 100;

func void changeName(struct personStruct person, string newName)
{
	person.name = newName;
	/* There's an example of a multi-line test case. 
}
with test: {
changeName(person, "newName");
person.name == "newName";
} using: { personStruct person = personStruct("name", 18)};


func int gcd(int a, int b)
{

	if (a == b) {
		return a;
	} else if (a > b) {
		gcd(a-b, b);
	} else {
		gcd(a, b-a);
	}

}
with test: gcd(10, 5) == 5;

int num1 = 25;
int num2 = 30;
int num3;

num3 = gcd(num1, num2);

print("Result %d\n" , num3);

\end{lstlisting}



\end{document} 