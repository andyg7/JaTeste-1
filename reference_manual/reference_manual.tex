\documentclass{article} 
\usepackage{url, graphicx}
\usepackage[margin=1in]{geometry}
\usepackage{textcomp}
\usepackage{algpseudocode}
\usepackage{algorithm}
\usepackage{titling}
\usepackage{amsmath}
\usepackage{amssymb}
\usepackage{amsthm}
\usepackage{fancyvrb}
\usepackage{listings} % for code highlighting/formatting

\usepackage{color} %defining colors for syntax highlighting
\definecolor{syntaxBlue}{RGB}{42,0.0,255}
\definecolor{syntaxGreen}{RGB}{63,127,95}
\definecolor{syntaxPurple}{RGB}{127,0,85}
\definecolor{syntaxCyan}{RGB}{0,155,155}
\definecolor{syntaxGreyBg}{RGB}{220,220,220}

\lstdefinelanguage{JaTeste} %define the code highlighting/formatting
{
	% list of keywords
	morekeywords={
		func,
		with,
		test,
		if,
		else,
		while,
		for,
		return,
		using,
		import
	},
	sensitive=true, % keywords ARE case-sensitive
	morecomment=[s]{/*}{*/}, % /* and */ delimit comments
	morestring=[b]" % string's MUST be in double quotes
}
\lstset{
	language={JaTeste}, % tell listings package to use the JaTeste language spec
	basicstyle=\small\ttfamily, % Global Code Style
	tabsize=2, % number of spaces indented when discovering a tab 
	columns=fixed, % make all characters equal width
	keepspaces=true, % does not ignore spaces to fit width, convert tabs to spaces
	showstringspaces=false, % lets spaces in strings appear as real spaces
	breaklines=true, % wrap lines if they don't fit
	frame=trbl, % draw a frame at the top, right, left and bottom of the listing
	frameround=tttt, % make the frame round at all four corners
	framesep=4pt, % quarter circle size of the round corners
	numbers=left, % show line numbers at the left
	numberstyle=\tiny\ttfamily, % style of the line numbers
	commentstyle=\color{syntaxGreen},
	keywordstyle=\color{syntaxPurple},
	stringstyle=\color{syntaxBlue},
	emph={int,char,float,struct,string},
	emphstyle=\color{syntaxCyan},
	backgroundcolor=\color{syntaxGreyBg},
}

\title{PLT 4115 LRM: \textbf{JaTest\'{e}}}
\author{
	Andrew Grant\\
	\texttt{amg2215@columbia.edu}
	\and
	Jemma Losh\\
	\texttt{jal2285@columbia.edu}
	\and
	Jared Weiss\\
	\texttt{jbw2140@columbia.edu}
	\and
	Jake Weissman\\
	\texttt{jdw2159@columbia.edu}
}

\date{\today}


\begin{document}

\maketitle
\newpage

\tableofcontents
\newpage

\section{Introduction}
The goal of JaTest\'{e} is to design a language that promotes good coding practices - mainly as it relates to testing.  JaTest\'{e} will require the user to explicitly define test cases for any function that is written in order to compile and execute code.  This will ensure that no code goes untested and will increase the overall quality of programmer code written in our language.  The user will be required to provide some test cases for their code, and the language will also generate some important test cases for their code as well.  JaTest\'{e} is mostly a functional language with a syntax quite similar to C.  The details of our language usage is provided in the rest of the document.

\section{Lexical Conventions}
This chapter will describe how input code will be processed and how tokens will be generated.

\subsection{Identifiers}
% Specs on how to name variables, functions, data types, etc.
Identifiers are used to name a variable, a function, or other types of data.  An identifier can include all letters, digits, and the underscore character.  An identifier must start with either a letter or an underscore - it cannot start with a digit.  Capital letters will be treated differently from lower case letters.

\begin{Verbatim}[frame=single]
ID = "(['a'-'z' 'A'-'Z'] | '_') (['a'-'z' 'A'-'Z'] | ['0'-'9'] | '_')*"
\end{Verbatim}


\subsection{Keywords}
% Just a list of reserved keywords
Keywords are a set of words that serve a specific purpose in our language and may not be used by the programmer for any other reason.  The list of keywords the language recognizes and reserves is as follows: 

\texttt{int char float struct if else for while break continue with test using func return string print}

\subsection{Constants}
% How to define constants such as x = 5
Our language includes integer, character, real number, and string constants. They're defined in the following sections.

\subsubsection{Integer Constants}
% We should specify all ways you can define an integer
Integer constants are a sequence of digits. An integer is taken to be decimal. Here are some examples:

13

1357

23

987

\subsubsection{Character Constants}
% Same for character
Character constants hold a single character and are enclosed in single quotes. They are stored in a variable of type char. Character constants that are preceded with a backslash have special meaning. Here are some examples:

'a'

'$\backslash$n' newline

'$\backslash$t' tab


\subsubsection{Real Number Constants}
% Do we want to allow for only ints?  If yes, delete this section
Real number constants represent a floating point number. They are composed of a sequence of digits, representing the whole number portion, followed by a decimal and another sequence of digits, representing the fractional part. Here are some examples. The whole part or the fractional part may be omitted, but not both.

3.14

0.4

.7

8

\subsubsection{String Constants}
% How to define a string constant
Strings are a sequence of characters enclosed by double quotes. A String is treated like a character array. Here are some examples:

``Hello, world!"

``JaTeste"

\subsection{Operators}
% Just note they can be used, will be explained more later
Operators are special tokens such as multiply, equals, etc. that are applied to one or two operands.  Their use will be explained further in chapter 4.

\subsection{White Space}
Whitespace is considered to be a space, tab, or newline. It is used for token delimitation, but has no meaning otherwise. 

\begin{Verbatim}[frame=single]
WHITESPACE = "[' '  '\t' '\r' '\n']"
\end{Verbatim}

\subsection{Comments}
A comment is a sequence of characters beginning with a forward slash followed by an asterisk. It continues until it is ended with an asterisk followed by a forward slash. Comments are treated as whitespace. 
\begin{Verbatim}[frame=single]
COMMENT = "/\* [^ \*/]* \*/ "
\end{Verbatim}

\subsection{Separators}
Separators are used to separate tokens. Separators are single character tokens, except for whitespace which is a separator, but not a token. 
\begin{Verbatim}[frame=single]
'('	{ LPAREN }
')'	{ RPAREN }
'{'	{ LBRACE }
'}'	{ RBRACE }
';'	{ SEMI }
'['	{ LBRACKET }
']'	{ RBRACKET }
'.'	{ DOT }
','	{ COMMA }
\end{Verbatim}

\section{Data Types}
The data types in JaTeste can be classified into three categories: primitive types, structures, and arrays.

\subsection{Primitives}
% Define primitives and values they can hold
\subsubsection{Integer Types}
The integer data type is a 32 bit value that can hold whole numbers ranging from $-2,147,483,648 \text{ to } 2,147,483,647$. Keyword 'int' is required to declare a variable with this type. 
\begin{lstlisting}
int a = 10;
int b = a + 10;
\end{lstlisting}

\subsubsection{Character Type}
The character type is an 8 bit value that is used to hold a single character. The keyword ``char'' is used to declare a variable with this type. 
\begin{lstlisting}
char c = 'h';
char b = 'e';
char d = 'l';
\end{lstlisting}

\subsection{Structures}
% I.e. structs
The structure data type is a collection of primitive types and other structure data types. The keyword ``struct'' followed by the name of the struct is used to define structures. Curly braces are then used to define what the structure is made of. As an example, consider the following:

\subsubsection{Defining Structures}

\begin{lstlisting}

struct person = {
string name;
int age;
int height;
};


struct manager = {
struct person name;
int salary;
};


\end{lstlisting}
Here we have defined two structs, the first being of type ``struct person'' and the second of type ``struct manager''. 

\subsubsection{Initializing Structures}
To create a structure, the struct type is followed by a variable name. 
\begin{lstlisting}
struct manager yahoo_manager;
struct person sam;
\end{lstlisting}

Here, we create two variables yahoo\_manager and sam. The first is of type ``struct manager'', and the second is of type ``struct person''.

\subsubsection{Accessing Structure Members}

 To access structs and modify its variables, a period following by the variable name is used:

 \begin{lstlisting}
yahoo_manager.name = sam;
yahoo_manager.age = 45;
yahoo_manager.salary = 65000;
\end{lstlisting}

Ultimately, all structures are backed by some collection of primitives. For example, the first structure, ``struct manager'', is made up of another struct and an int. Since ``struct person'' is made up of two ints, ``struct manager'' is really just made up of three ints. 


\subsection{Arrays}
An array is a data structure that allows for the storage of one or more elements of the same data type consecutively in memory. Each element is stored at an index, and array indices begin at 0. This section will describe how to use Arrays.

\subsubsection{Defining Arrays}
An array is declared by specifying its data type, name, and size. The size must be positive. Here is an example of declaring an integer array of size 10:

\begin{Verbatim}[frame=single]
typ ID LBRACKET INT RBRACKET
\end{Verbatim}

 \begin{lstlisting}
int arr[5];
\end{lstlisting}

\subsubsection{Initializing Arrays}
An array can be initialized by declared and initialized in the same line, by listing the element values separated by commas and surrounded by brackets. Here is an example:

\begin{Verbatim}[frame=single]
ID ASSIGN LBRACE expr RBRACE
\end{Verbatim}

 \begin{lstlisting}
arr = { 0, 1, 2, 3, 4 };
\end{lstlisting}

It is not required to initialize all of the elements. Elements that are not initialized will have a default value of zero. For example in this array, arr[3] and arr[4] would have a value of 0:

 \begin{lstlisting}
int arr[5] = { 0, 1, 2 };
\end{lstlisting}

\subsubsection{Accessing Array Elements}
To access an element in an array, use the array name followed by the element index surrounded by square brackets. Here is an example that assigns the value 1 to the first element (at index 0) in the array:

\begin{Verbatim}[frame=single]
ID LBRACKET INT RBRACKET ASSIGN expr
\end{Verbatim}

 \begin{lstlisting}
arr[0] = 1;
\end{lstlisting}

JaTeste does not test for index out of bounds, so the following code would compile although it is incorrect. 

 \begin{lstlisting}
int arr[2];
arr[5] = 1;
\end{lstlisting}


\subsubsection{Multidimensional Arrays}
A multidimensional array is an array of arrays. To declare a multidimensional array, add another square bracket pair indicating the size for each dimension. 

\begin{Verbatim}[frame=single]
typ ID LBRACKET INT RBRACKET LBRACKET INT RBRACKET
\end{Verbatim}


Here is an example of declaring and initializing a 2D array that holds 3 elements per dimension:

\begin{Verbatim}[frame=single]
typ ID LBRACKET INT RBRACKET LBRACKET INT RBRACKET ASSIGN LBRACE expr RBRACE
\end{Verbatim}

 \begin{lstlisting}
int a[2][3] = { {0, 1, 2}, {3, 4, 5} };
\end{lstlisting}

To access an element in a multidimensional array, specify the desired index of all dimensions. For example:

\begin{Verbatim}[frame=single]
ID LBRACKET INT RBRACKET LBRACKET INT RBRACKET ASSIGN expr
\end{Verbatim}

 \begin{lstlisting}
a[0][1] = 1;
\end{lstlisting}

\subsubsection{Arrays of Structures}
An array of structures is an array that holds elements of a structure type. Here is an example of declaring a struct array, that holds three struct inventory elements. 

\begin{Verbatim}[frame=single]
STRUCT STRING_LITERAL ID LBRACKET INT RBRACKET
\end{Verbatim}

 \begin{lstlisting}
struct inventory {
	int part;
	float cost;
};

struct inventory in_array[3];

\end{lstlisting}

To initialize the elements in a struct array, brackets are used to surround the members associated with each struct element. Here is an example:

\begin{Verbatim}[frame=single]
STRUCT STRING_LITERAL ID LBRACKET INT RBRACKET ASSIGN LBRACE expr RBRACE
\end{Verbatim}

 \begin{lstlisting}
struct inventory in_array[3] = { {1, 1.0}, {2, 2.0}, {3, 3.0} };
\end{lstlisting}

To access the members of the elements of a struct array, the member access operator is used, with the array name and element number preceding it. Here is an example:

\begin{Verbatim}[frame=single]
ID LBRACKET INT RBRACKET DOT ID ASSIGN expr
\end{Verbatim}

 \begin{lstlisting}
in_array[0].part = 1;
in_array[0].cost = 1.0;
\end{lstlisting}

\section{Expressions and Operators}

\subsection{Expressions}
An expression is a collection of one or more operands and zero or more operators that can be evaluated to produce a value.  A function that returns a value can be an operand as part of an expression.  Additionally, parenthesis can be used to group smaller expressions together as part of a larger expression.  A semicolon terminates an expression.  Some examples of expressions include:
\begin{lstlisting}
35 - 6;
foo(42) * 10;
8 - (9 / (2 + 1) );
\end{lstlisting}

\subsection{Assignment Operators}
% =, +=, -=, etc
Assignment can be used to assign the value of an expression on the right side to a named variable on the left hand side of the equals operator.  The left hand side can either be a named variable that has already been declared or a named variable that is being declared and initialized in this assignment.  Examples include:
\begin{lstlisting}
int x = 5;
float y;
y = 9.9;
\end{lstlisting}

Additionally, the following operators can also be used for variations of assignment:

\begin{itemize}
\item \texttt{+=} increments the left hand side by the result of the right hand side
\item \texttt{-=} decrements the left hand side by the result of the right hand side
\end{itemize}

\subsection{Incrementing and Decrementing}
% ++, --, etc.
This can be done using the \texttt{++} operator to increment and the \texttt{--} operator to decrement a value.  If the operator is placed before a value it will be incremented / decremented first, then it will be evaluated.  If the operator is placed following a value, it will be evaluated with its original value and then incremented / decremented.

\subsection{Arithmetic Operators}
% +, -, ...
\begin{itemize}
\item \texttt{+} can be used for addition
\item \texttt{-} can be used for subtraction (on two operands) and negation (on one operand)
\item \texttt{*} can be used for multiplication
\item \texttt{/} can be used for division
\item \texttt{$\wedge$} can be used for exponents
\item \texttt{$\%$} can be used for modular division
\end{itemize}

\subsection{Comparison Operators}
% ==, >, <, etc.
\begin{itemize}
\item \texttt{==} can be used to evaluate equality
\item \texttt{!=} can be used to evaluate inequality
\item \texttt{>} can be used to evaluate is the left greater than the right
\item \texttt{>=} can be used to evaluate is the left greater than or equal to the right
\item \texttt{<} can be used to evaluate is the left less than the right
\item \texttt{<=} can be used to evaluate is the left less than or equal to the right
\end{itemize}

\subsection{Logical Operators}
% &&, ||
\begin{itemize}
\item \texttt{!} can be used to evaluate the negation of one expression
\item \texttt{$\&\&$} can be used to evaluate logical and
\item \texttt{$\vert\vert$} can be used to evaluate logical or
\end{itemize}

\subsection{Operator Precedence}

\subsection{Order of Evaluation}
% ++ vs * and such

\section{Statements}
Statements include: \texttt{if, while, for, return} as explained in the following sections

\subsection{If Statement}
% explain if, else if, else
The if, else if, else construct will work as expected in other languages.

\begin{lstlisting}
if (x == 42) {
	print("Gotcha");
}
else if (x > 42) {
	print("Sorry, too big");
}
else {
	print("I\'ll allow it");
}
\end{lstlisting}

\subsection{While Statement}
The while statement will evaluate in a loop as long as the specified condition in the while statement is true.

\begin{lstlisting}
/* Below code prints "Hey there" 10 times */
int x = 0;
while (x < 10) {
	print("Hey there");
	x++;
}
\end{lstlisting}

\subsection{For Statement}
The for condition will also run in a loop so long as the condition specified in the for statement is true.  The expectation for a for statement is as follows:

\texttt{for ( <initial state>; <test condition>; <step forward> )}

Examples are as follows:

\begin{lstlisting}
/* 	This will run as long as i is less than 100
	i will be incremented on each iteration of the loop */
for (int i = 0; i < 100; i++) {
	/* do something */
}

/* i can also be declared or initialized outside of the for loop */
int i;
for (i = 0; i < 100; i += 2) {
	/* code block */
}
\end{lstlisting}

\subsection{Code Blocks}
% Code within braces
Blocks are code that is contained within a pair of brackets, \texttt{\{ code \}}, that gets executed within a statement.  For example, any code blocks that follow an \texttt{if} statement will get executed if the \texttt{if} condition is evaluated as true:

\begin{lstlisting}
int x = 42;
if (x == 42) {
	/* the following three lines are executed */
	print("Hey");
	x++;
	print("Bye");
}
\end{lstlisting}

\subsection{Return Statement}
The \texttt{return} statement is used to exit out of a function and return a value.  The return value must be the same type that is specified by the function deceleration.  Return can be used as follows:

\begin{lstlisting}
/* The function trivially returns the input int value */
func int someValue(int x) {
	return x;
}
\end{lstlisting}

\section{Functions}
Functions allow you to group snippets of code together that can subsequently be called from other parts of your program, depending on scope. Functions are global, unless they are prepended with the keyword ``private''. While not necessary, it is encouraged that you declare functions before defining them. Functions are usually declared at the top of the file they're defined in. Functions that aren't declared can only be called after they have been defined. 

\subsection{Function Declarations}

The keyword ``func'' is used to declare a function. A return type is also required using keyword ``return''; if your function doesn't return anything then use keyword ``void'' instead.  Functions are declared with or without parameters; if parameters are used, their types must be specified. A function can be defined with multiple, different parameters. Though a function can only have one return type, it can also be any data type. 

\begin{lstlisting}
func int add(int a, int b); /* this functions has two int parameters as input and returns an int */
func void say_hi(); /* this function doesn't return anything nor takes any parameters */
func int isSam(string name, int age); /* this functions has two input parameters, one of type string and one of type int */

\end{lstlisting}

\subsection{Function Definitions}
Function definitions contain the instructions to be performed when that function is called. The first part of the syntax is similar to how you declare functions; but curly brackets are used to define what the function actually does. For example, 

\begin{lstlisting}
func int add(int a, int b); /* declaration */

func int add(int x, int y) /* definition */
{
return x + y;
}
\end{lstlisting}

This snippet of code first declares add, and then defines it. Declaring before defining is best practice. Importantly, functions can \emph{not} reference global variables; that is, the only variables they can act on are formal parameters and local variables. For example:

\begin{lstlisting}
func int add_to_a(int x); /* declaration */
int a = 10;
func int add_to_a(int x) /* definition */
{
return x + a; /* this is NOT allowed */
}
\end{lstlisting}

This code is no good because it relies on global variable ``a''. Functions can only reference formal parameters and/or local variables.

\subsection{Calling Functions}

A function is called using the name of the function along with any parameters it requires. You \emph{must} supply a function with the parameters it expects. For example, the following will not work:

\begin{lstlisting}
func int add(int a, int b); /* declaration */

func int add(int x, int y) /* definition */
{
return x + y;
}

add(); /* this is wrong and will not compile because add expects two ints as parameters */

\end{lstlisting}

Functions can only be called after they have been declared and/or defined. Functions are first class objects and so can be used anywhere a normal data type can be used. Of course, a function's return type must be compatible with the context it's being used in. For example, a function that returns a char cannot be used as an actual parameter to a function that expects an int.  Consider the following:

\begin{lstlisting}
func int add_int(int a, int b); /* declaration */

func int add_int(int x, int y) /* definition */
{
return x + y;
}

func float add_float(float x, float y)
{
	return x + y;
}

func int subtract(int x, int y)
{
	return x - y;
}

int answer = subtract(add(10,10), 10); /* this is ok */
int answer2 = subtract(add_float(10.0,10.0), 10); /* this is NOT ok because subtract expects its first parameter to be an int while add_float returns a float */

\end{lstlisting}

\subsection{Function Parameters}
Formal parameters can be any data type. Furthermore, they need not be of the same type. For example, the following is syntactically fine:

\begin{lstlisting}
func void speak(int age, string name)
{
	print_string ("My name is" + name + " and I am "  + age);
}
\end{lstlisting}
While functions may be defined with multiple formal parameters, that number must be fixed. That is, functions cannot accept a variable number of arguments. 
\\
TODO: Are we passing by value or reference?

% \subsection{Main Function}
% Do we want a main function?

\subsection{Recursive Functions}

Functions can be used recursively. Each recursive call results in the creation of a new stack and new set of local variables. It is up to the programmer to prevent infinite loops. 

\subsection{Function Test Cases}
Functions can be appended with test cases directly in the source code. Most importantly, the test cases will be compiled into a separate (executable) file. The keyword ``with test'' is used to define a test case as illustrated here:

\begin{lstlisting}
func int add(int a, int b); /* declaration */

func int add(int x, int y) /* definition */
{
return x + y;
}
with test {
	add(1,2) == 3;
	add(-1, 1) == 0;
}
with test {
	add(0,0) <= 0;
	add(0,0) >= 0; 
}

\end{lstlisting}
Test cases contain a set of boolean expressions. Multiple boolean expressions can be defined, they just must be separated with semi-colons. As shown above, you can define separate test cases one after another too. \\ 
Snippets of code can also be used to set up a given test case's enviornment using the ``using'' keyword. That is, ``using'' is used to define code that is executed right before the test case is run. Consider the following:

\begin{lstlisting}
func void changeAge(struct person temp_person, int age)
{
temp_person.age = age;
}
with test {
	sam.age == 11;
}
using {
struct person sam;
sam.age = 10;
changeAge(sam, 11);
}
\end{lstlisting}
``using'' is used to create a struct and then call function changeAge; it is setting up the enviornment for it's corresponding test. Variables defined in the ``using'' section of code can safely be referenced in its corresponding test case as shown. Basically, the code in the ``using'' section is executed right before the boolean expressions are evaluated and tested. \\
The ``using'' section is optional. As a result some test cases may contain ``using'' sections and others might not. As per convention, each ``using'' section will match up with its closest test case. For, example:

\begin{lstlisting}

func int add(int x, int y) /* definition */
{
return x + y;
}
with test { /*  variables a, b defined below are NOT in this test case's scope*/
	add(1,2) == 3;
	add(-1, 1) == 0;
}
with test { /* variables a and b ARE in this test case's scope */
	add(a, b) == 20;
}
using {
int a = 10;
int b = 10;
}

\end{lstlisting}

As explained in the comments, the ``using'' section is matched up with the second test case.
\\
Test cases are compiled into a separate program which can subsequently be run. The program will run all test cases and output appropriate information. 

\section{Program Structure and Scope}

\subsection{Program Structure}

\subsection{Scope}

\section{A Sample Program}

\end{document}