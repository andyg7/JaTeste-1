\documentclass{article}[10pt]
\usepackage{url, graphicx}
\usepackage[margin=1in]{geometry}
\usepackage{textcomp}
\usepackage{algpseudocode}
\usepackage{algorithm}
\usepackage{titling}
\usepackage{amsmath}
\usepackage{amssymb}
\usepackage{amsthm}
\usepackage{float}
\usepackage{cite}
\usepackage{listings}


\title{HW3 CS 4115}
\author{Andrew Grant, amg2215@columbia.edu}
\date{}
\begin{document}
\maketitle


\section{}
\begin{enumerate}
\item[(a)] Let any integer $i = a_3 a_2 a_1 a_0$ where $a_i$ is some byte \\
  \begin{tabular}{| c | c  | c | c | |   c | }
    \hline
    $a_3$ & $a_2$ & $a_1$ & $a_0$ & a[0][0]  \\  \hline
    $a_3$ & $a_2$ & $a_1$ & $a_0$ & a[0][1]  \\  \hline
     $a_3$ & $a_2$ & $a_1$ & $a_0$ & a[0][2]  \\  \hline
     $a_3$ & $a_2$ & $a_1$ & $a_0$ & a[1][0]  \\  \hline
     $a_3$ & $a_2$ & $a_1$ & $a_0$ & a[1][1]  \\  \hline
     $a_3$ & $a_2$ & $a_1$ & $a_0$ & a[1][2]  \\  \hline
\end{tabular}
  
\item[(b)]
address = $(i \times 12) + (j \times 4)$

\item[(c)]
Assembly code: \\
\begin{lstlisting}
       .file   "main.c"
        .text
        .globl  main
        .type   main, @function
main:
.LFB0:
        .cfi_startproc
        movl    $0, a(%rip)
        movl    $4, a+4(%rip)
        movl    $8, a+8(%rip)
        movl    $12, a+12(%rip)
        movl    $16, a+16(%rip)
        movl    $20, a+20(%rip)
        movl    $0, %eax
        ret
        .cfi_endproc
.LFE0:
        .size   main, .-main
        .comm   a,24,16
        .ident  "GCC: (Ubuntu 4.8.4-2ubuntu1~14.04.1) 4.8.4"
        .section        .note.GNU-stack,"",@progbits
\end{lstlisting}
Corresponding C code: \\
\begin{lstlisting}
int a[2][3];
int main()
{       
        int i;  
        int j;
        for (i = 0; i < 2; i++) {
                for (j = 0; j < 3; j++) {
                        a[i][j] = (12 * i) + (4 * j);
                }
        }
        
        return 0;
}
\end{lstlisting}
Basically, I put the value $(12 \times i) + (4 \times j)$ into $a[i][j]$ in a for-loop. As can be seen, at each step in the for-loop, the value of $(12 \times i) + (4 \times j)$ is precisely the offset from the base address of the label a (the base address of the 2D array) that $(12 \times i) + (4 \times j)$ is stored into; thus $(12 \times i) + (4 \times j)$ does indeed correspond to the address of $a[i][j]$ 
  
\end{enumerate}

\end{document} 