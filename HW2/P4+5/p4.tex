\documentclass{article} 
\usepackage{url, graphicx}
\usepackage[margin=1in]{geometry}
\usepackage{textcomp}
\usepackage{algpseudocode}
\usepackage{algorithm}
\usepackage{titling}
\usepackage{amsmath}
\usepackage{amssymb}
\usepackage{amsthm}
\usepackage{listings}
\usepackage{comment}

\usepackage{tikz}
\usepackage{pgfplots}
\pgfplotsset{width=8cm,compat=1.9}
\usetikzlibrary{pgfplots.dateplot}
\usepackage{pgfplotstable}
\usepackage{filecontents}

\usepackage{qtree}




\begin{document}


%Section 1
\section{}
\begin{enumerate}
\item[a)]

$S \Rightarrow_{rm}  \underline{( L )} \Rightarrow_{rm} ( \underline{L , S} ) \Rightarrow_{rm} ( L , \underline{( L )} ) \Rightarrow_{rm} ( L , ( \underline{L , S} ) ) \Rightarrow_{rm} ( L , ( L , \underline{a} ) ) \Rightarrow_{rm} ( L , ( \underline{S} , a ) ) \Rightarrow_{rm} ( L , ( \underline{a} , a ) )  \Rightarrow_{rm}( \underline{S} , ( a , a ) ) \Rightarrow_{rm} ( \underline{a} , ( a , a ) )$

\item[b)]
\begin{tabular}{| r | c | l |}
  \hline
  & (a, (a,a)) & shift \\
   ( & a, (a,a)) & shift \\
   ( a & , (a,a)) & reduce \\
   ( S & , (a,a)) & reduce \\
   ( L & , (a,a)) & shift \\
   ( L , & (a,a)) & shift \\
   ( L, ( & a,a)) & shift \\
   ( L , ( a & ,a)) & reduce \\
   ( L , ( S & , a)) & reduce \\
   ( L , ( L & , a)) & shift \\
   ( L , ( L , & a )) & shift \\
   ( L , ( L , a & )) & reduce \\
   ( L , ( L , S & )) & reduce \\
   ( L , ( L & )) & shift \\
   ( L , ( L ) & ) & reduce \\
   ( L , S & ) & shift \\
   ( L , S ) & & reduce \\
   ( L ) & & reduce \\
   S & & accept \\

  \hline \hline
  Stack & Input & Move \\
  \hline
\end{tabular}

\item[c)]

\end{enumerate}

\section{}

State 2 has a shift/reduce conflict. When the next input token is else, the parser doesn't know whether it should reduce T according to rule 3 or shift else onto the stack. This is an example of the dangling else problem. For clarity, the state with a conflict is: \\

\begin{tabular}{| l |}
  \hline
  $\text{S} \rightarrow \text{\textbf{if} S . T}$  \\
  $\text{T} \rightarrow \text{. \textbf{else} S}$  \\
  $\text{T} \rightarrow \text{.}$  \\
  \hline
\end{tabular}

\end{document} 

